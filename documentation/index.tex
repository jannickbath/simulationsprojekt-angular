\documentclass[hidelinks,12pt]{article}
\usepackage{enumitem}
\usepackage{titlesec}
\usepackage[utf8]{inputenc}
\usepackage[left=2.5cm, right=2cm, top=2cm, bottom=2cm]{geometry}
\usepackage{fancyhdr}
\usepackage{hyperref}
\usepackage[german]{babel}

% Header settings
\pagestyle{fancy}
\fancyhf{}
\rhead{Autor: Jannick Bath / FIN 11}
\lhead{Simulationsprojekt Dokumentation}
\rfoot{\thepage}
\lfoot{Datum: \today}

\begin{document}

\begin{titlepage}
    \centering
    \vspace*{4cm} % Adjust the vertical space as needed

    \huge\bfseries Simulationsprojekt Dokumentation\par
    \vspace{1.5cm} % Adjust the vertical space as needed

    \Large 22.01.2024\par
    \vspace{1.5cm} % Adjust the vertical space as needed

    \large Jannick Bath\par
    \vspace{4cm} % Adjust the vertical space as needed

    \vfill

\end{titlepage}

\newpage

\tableofcontents % Hier wird das Inhaltsverzeichnis eingefügt

\newpage

\section{Projektziel}
Das Hauptziel dieses Projekts besteht darin, ein herausforderndes Singleplayer-Rennspiel zu entwickeln, bei dem die Spieler ihre Fahrzeuge durch das Schreiben von Texten steuern. Die Schreibgeschwindigkeit beeinflusst die Fortbewegungsgeschwindigkeit des Fahrzeugs direkt. Zusätzlich soll ein spezieller Zufallsfaktor in Form von zufälligen Geschwindigkeitsänderungen für Bots und verschiedenen Items, wie Barrieren, das Spielerlebnis variabel gestalten.

\section{Art des Projektes}
Das Projekt fällt in die Kategorie der Unterhaltungsspiele und kombiniert Elemente des Rennspiels mit einer Schreibmechanik. Es ist als Singleplayer-Erfahrung konzipiert, bei der Spieler gegen Bots antreten, um als Erster die Ziellinie zu überqueren.

\section{Betriebliche Umfeldbeschreibung}

\subsection{Auftragnehmer}
Ich agiere als Auftragnehmer für dieses Projekt im Rahmen meiner schulischen Ausbildung zum Fachinformatiker für Anwendungsentwicklung. Meine Expertise umfasst den Einsatz des Contao Frameworks, PHP, Typescript, JavaScript, SCSS, CSS und HTML. Ich experimentiere auch gerne mit Frontend-Frameworks wie React oder Angular. Dieses Simulationsprojekt ist ein integraler Bestandteil meiner schulischen Ausbildung und bietet eine praxisnahe Anwendung meiner erworbenen Fähigkeiten.

\subsection{Auftraggeber}
Die Berufsschule BS-Technik Rostock fungiert als Auftraggeber dieses Projekts im Rahmen meiner Ausbildung. Die Anforderungen wurden in Absprache mit der Berufsschule festgelegt, wobei der Fokus auf den folgenden Kriterien liegt:

\begin{itemize}
    \item Die Oberfläche sollte mindestens 5 Eingabe-Parameter aufweisen.
    \item Es sollten zwei Zufallsparameter implementiert werden, einmal basierend auf der Gaußschen Normalverteilung und einmal auf der Normalverteilung.
    \item Es sollten drei verschiedene Stufen der Simulationsgeschwindigkeit vorhanden sein: langsam, mittel, schnell.
    \item Mindestens eine Animation muss in der Simulation integriert werden. Die Simulation sollte zeitabhängig sein.
\end{itemize}

\section{Ausgangssituation}
Die Berufsschule BS-Technik Rostock beauftragt im Rahmen der Ausbildung zum Fachinformatiker für Anwendungsentwicklung die Entwicklung einer Simulation. Die genauen Anforderungen sind bereits unter dem Punkt Auftraggeber formuliert.

\section{Analyse vom Ist-Zustand}
Der aktuelle Ist-Zustand zeigt, dass bisher keine vorhandene Simulation oder vergleichbare Anwendung mit den spezifischen Anforderungen der Berufsschule BS-Technik Rostock existiert. Dies eröffnet die Möglichkeit, eine Simulation von Grund auf zu entwickeln, die den gewünschten Kriterien entspricht.

\section{Soll-Konzept}

\subsection{Architektur und Design}
Für die Architektur des Projekts habe ich die Umsetzung als React-Typescript-Projekt vorgesehen, wobei das Zustandsmanagement durch die Verwendung der Bibliothek "Zustand" realisiert wird. Die Architektur folgt einer objektorientierten Herangehensweise, um die Prinzipien der Clean-Code-Ideologie bestmöglich umzusetzen.

Das Design des Projekts konzentriert sich auf eine benutzerfreundliche Gestaltung der Benutzeroberfläche. Dabei liegt der Fokus darauf, die Eingabe-Parameter, Zufallsparameter, Geschwindigkeitsstufen, Animationen und zeitliche Abhängigkeit übersichtlich und intuitiv zu präsentieren. Die Visualisierung der Spielmechaniken erfolgt klar und ansprechend, um eine gute Benutzererfahrung zu gewährleisten.

Der Ablaufplan für die Umsetzung sieht vor, zunächst das UI zu designen und zu entwickeln. Anschließend wird die Implementierung der Grundmechaniken eines textbasierten Rennspiels folgen. Zum Abschluss ist geplant, den Zufallsfaktor durch die Integration bestimmter Items einzuführen.

Des Weiteren ist die Implementierung von Black-Box-Tests vorgesehen, die mithilfe der Bibliothek "Cypress" durchgeführt werden. Dies stellt sicher, dass die Funktionalitäten des Projekts in verschiedenen Szenarien erfolgreich getestet werden.

\subsection{Auswahl der Technologien}
Für die Auswahl der Technologien habe ich mich auf bewährte Werkzeuge konzentriert, die die Effizienz und Qualität des Projekts sicherstellen.

\subsubsection{Frontend}
Die Verwendung von React in Verbindung mit Typescript bietet eine robuste Basis für die Frontend-Entwicklung. Das Zustandsmanagement erfolgt durch die Bibliothek "Zustand", was eine effektive Verwaltung des Anwendungsstatus ermöglicht.

\subsubsection{Backend}
Es war keine Umsetzung eines Backends angedacht.

\subsubsection{Testing}
Für das Testen habe ich auf Black-Box-Tests vorgesehen, die mithilfe der Bibliothek "Cypress" durchgeführt werden sollen. Dies ermöglicht umfassende Tests der Benutzeroberfläche und der Anwendungslogik in verschiedenen Szenarien.

Die Entscheidungen bei der Auswahl dieser Technologien basieren auf deren Zuverlässigkeit, Leistungsfähigkeit und ihrer Anpassungsfähigkeit an die Anforderungen des Projekts.

\subsection{Datenmodellierung}
Die Datenmodellierung erfolgt gemäß der Vorgaben von "Zustand", wodurch eine temporäre Speicherung der Daten in klar abgegrenzten Slices erfolgt. Diese Slices ermöglichen eine klare Strukturierung und Abgrenzung der Daten im Bezug auf verschiedene Komponenten der Simulation.

Da das Projekt kein Backend vorsieht, werden keine persistenten Daten verwendet. Die Datenverwaltung konzentriert sich auf die effiziente Nutzung des "Zustand"-Managements, um eine konsistente und reaktive Verwaltung der Daten während der Laufzeit der Anwendung zu gewährleisten.

Die objektorientierte Herangehensweise wird dabei beibehalten, um die Struktur und Wartbarkeit der Datenmodelle zu optimieren.

\subsection{Implementierung der Simulation}
Die Implementierung der Simulation folgt einem strukturierten Ablaufplan, der sicherstellt, dass die Funktionalitäten schrittweise und konsistent entwickelt werden.

\subsubsection{UI-Design und -Entwicklung}
Die erste Priorität liegt auf der Gestaltung und Entwicklung einer benutzerfreundlichen Benutzeroberfläche. Hierbei werden die Eingabe-Parameter, Zufallsparameter, Geschwindigkeitsstufen, Animationen und die zeitliche Abhängigkeit in die Oberfläche integriert. Die klare und intuitive Darstellung dieser Elemente steht im Mittelpunkt dieses Schrittes.

\subsubsection{Grundmechaniken des Rennspiels}
Im zweiten Schritt erfolgt die Programmierung der grundlegenden Mechaniken des textbasierten Rennspiels. Hierbei wird die Steuerung durch Tastatureingaben implementiert, um die Fortbewegung des virtuellen Fahrzeugs zu ermöglichen. Der Fokus liegt auf der Umsetzung einer realistischen und ansprechenden Spielerfahrung.

\subsubsection{Integration des Zufallsfaktors}
Der dritte Schritt beinhaltet die Einbindung von Items, die den Zufallsfaktor in die Simulation bringen. Besonderes Augenmerk wird dabei auf Barrieren als Items gelegt, die den Fortschritt des Spielers verlangsamen. Die Implementierung erfolgt so, dass der Zufallsfaktor das Spielerlebnis bereichert und gleichzeitig herausfordernde Elemente hinzufügt.

Die Umsetzung erfolgt iterativ, wobei nach jedem Schritt eine Überprüfung auf Funktionalität und Benutzerfreundlichkeit erfolgt. Dadurch wird sichergestellt, dass die Simulation schrittweise und konsistent entwickelt wird.

\subsection{Benutzerinteraktion und Eingabe-Parameter}
Die Umsetzung erfolgt mit dem Fokus darauf, die Nutzerfreundlichkeit zu maximieren und dem Spieler eine transparente Steuerung über alle relevanten Spielaspekte zu bieten.

\subsubsection{Benutzerinteraktion}
Die zentrale Schnittstelle des Spiels ist die Textbox, in der der Nutzer den Text eingibt, um sein Fahrzeug voranzubringen. Dabei wird direktes Feedback in Echtzeit gegeben, ob das auf der Tastatur getippte Zeichen korrekt ist oder nicht. Dies ermöglicht dem Spieler eine unmittelbare Kontrolle über die Geschwindigkeit seines Fahrzeugs.

\subsubsection{Rennstrecken-Ansicht}
Unterhalb der Textbox befindet sich die Rennstrecken-Ansicht, die einen Überblick über das eigene Fahrzeug und die anderen Fahrzeuge der Bots bietet. Hier kann der Spieler den Fortschritt zu Ziellinie verfolgen. Die Fahrzeuge werden animiert, um Veränderungen im Fortschritt anschaulich darzustellen und die Spielatmosphäre zu intensivieren.

\subsubsection{Item Boxen}
Es gibt zwei weitere Boxen, die mithilfe von Icons ihre Funktionen anzeigen. Eine Box kann mit der "Alt"-Taste aktiviert werden, während die andere mit der "AltGraph"-Taste aktiviert wird. Die erste Box ermöglicht die Auswahl eines gegnerischen Fahrzeugs, während die zweite Box die Möglichkeit bietet, ein Item zu diesem Fahrzeug zu senden. Die klare visuelle Darstellung durch Icons sorgt für eine intuitive Bedienung dieser Funktionen.

\subsubsection{Geschwindigkeitsregler}
Ein zusätzlicher Regler wird implementiert, der es dem Spieler ermöglicht, die Simulationsgeschwindigkeit nach seinen Präferenzen einzustellen. Dies bietet eine flexible Anpassung des Spieltempos.

\subsubsection{Uhr}
Zur Verbesserung der Immersion wird eine Uhr oder ein Timer integriert, der die aktuelle Simulationszeit anzeigt. Dies gibt dem Spieler einen klaren Überblick darüber, wie lange die Simulation bereits läuft.

\subsection{Teststrategie}
Die folgende Teststrategie zielt darauf ab, eine robuste und fehlerfreie Simulation zu gewährleisten.

\subsubsection{Unit-Tests für einzelne Komponenten}
Es werden spezifische Unit-Tests entwickelt, um die Funktionalitäten einzelner Komponenten, wie die Textbox, Rennstrecken-Ansicht und Items, auf Herz und Nieren zu prüfen. Dies stellt sicher, dass jede Komponente in Isolation korrekt funktioniert.

\subsubsection{Integrationstests für die Gesamtsimulation}
Die Integrationstests fokussieren sich darauf, die Wechselwirkungen zwischen den verschiedenen Komponenten sicherzustellen. Dabei wird die Gesamtsimulation auf verschiedene Szenarien und Nutzereingaben getestet, um eine reibungslose Zusammenarbeit aller Elemente zu gewährleisten.

\subsection{Dokumentation}
Die Dokumentation erfolgt durch sorgfältige Kommentierung des Codes, wobei besonderes Augenmerk auf die Erläuterung von Funktionen, Strukturen und Implementierungsdetails gelegt wird.

Die Kommentierung zielt darauf ab, die Klarheit und Verständlichkeit der Codebasis zu verbessern.

\section{Ressourcenplanung}

\subsection{Personalplanung}
Die alleinige Umsetzung des Projekts erfolgt im Rahmen meiner Ausbildung als Fachinformatiker für Anwendungsentwicklung. Alle Entwicklungsaktivitäten, von der Planung über die Umsetzung bis zur Dokumentation, fallen in meinen Verantwortungsbereich.

\subsection{Sachmittelplanung}
Die genutzten Tools für die Entwicklung, wie React, TypeScript, Zustand und Cypress, sind kostenfrei verfügbar und erfordern keine zusätzlichen finanziellen Aufwendungen. Diese Auswahl gewährleistet eine kosteneffiziente Entwicklung.

\subsection{Terminplanung}
Die Terminplanung unterteilt sich in verschiedene Phasen: Planung (15%), Umsetzung (50%) und Dokumentation (35%). Die zeitliche Verteilung ermöglicht eine gezielte Fokussierung auf jede Phase des Projekts. Von den insgesamt verfügbaren 80 Stunden entfallen demnach 12 Stunden auf die Planung, 40 Stunden auf die Umsetzung und 28 Stunden auf die Dokumentation.

\subsection{Kostenplanung}
Da alle genutzten Tools kostenfrei sind und keine externen Ressourcen eingebunden werden, entstehen keine direkten Kosten im Rahmen der Entwicklung. Die Kostenplanung beschränkt sich somit auf die effiziente Nutzung der verfügbaren Zeit.

\section{Ablaufplanung}

\subsection{Planungsphase}

\begin{itemize}
    \item Definition der Projektziele und Anforderungen.
    \item Auswahl der geeigneten Tools und Technologien.
    \item Festlegung der Architektur und Designgrundlagen.
\end{itemize}

\subsection{Umsetzungsphase}

\begin{itemize}
    \item Design und Entwicklung der Benutzeroberfläche (UI).
    \item Implementierung der grundlegenden Mechaniken des Text-basierten Rennspiels.
    \item Integration von Simulationsgeschwindigkeiten und Animationen.
    \item Einbindung von Items und Zufallsfaktoren.
\end{itemize}

\subsection{Dokumentationsphase}

\begin{itemize}
    \item Fortlaufende Dokumentation der Entwicklungsfortschritte.
    \item Kommentierung des Codes für eine klare Nachvollziehbarkeit.
    \item Erstellung der Projektdokumentation im Markdown-Format.
    \item Überarbeitung und Verbesserung der Dokumentation nach Bedarf.
\end{itemize}

\section{Darstellung der Prozessketten}

\subsection{Benutzeroberfläche (UI)}
Die Nutzerschnittstelle bildet den Ausgangspunkt der Prozessketten. Hier interagiert der Nutzer mit der Anwendung, gibt Texte ein und erhält Feedback zum Fortschritt. Die Prozesskette beginnt mit der Erfassung der Nutzereingabe in der Textbox und setzt sich fort mit der Verarbeitung dieser Eingabe, um die Simulation entsprechend zu beeinflussen.

\subsection{Textbasiertes Rennspiel}
Die Mechaniken des textbasierten Rennspiels bilden den zentralen Prozess. Hier werden die eingegebenen Texte verarbeitet, die Simulationsgeschwindigkeiten berechnet und die Animationen der Fahrzeuge gesteuert. Die Schnittstellen zwischen der Benutzeroberfläche und dem Spielprozess sind entscheidend, um eine reibungslose Interaktion zu gewährleisten.

\subsection{Zufallsfaktoren und Items}
Die Integration von Zufallsfaktoren und Items führt zu weiteren Prozessketten. Hier werden die verschiedenen Elemente, die das Spiel dynamisch gestalten, in den Ablauf eingebunden. Die Schnittstellen zwischen dem textbasierten Rennspiel und den Zufallsfaktoren sind dabei essenziell, um einen organischen und unterhaltsamen Spielverlauf zu ermöglichen.

\subsection{Dokumentation und Code-Kommentierung}
Die Prozessketten setzen sich bis zur Dokumentation und Code-Kommentierung fort. Hier wird der Entwicklungsprozess transparent dokumentiert, und die Schnittstellen zwischen dem Code und der projektspezifischen Dokumentation gewährleisten eine klare Nachvollziehbarkeit.

\section{Vorgehensweise}
Die Entwicklung des Simulationsprojekts orientierte sich an einem Wasserfall-Vorgehensmodell. In der Planungsphase wurden die Projektziele konkretisiert, indem die Idee eines textbasierten Rennspiels formuliert wurde, das sowohl die Anforderungen des Simulationsprojekts erfüllt als auch unterhaltsamen Spielspaß bietet.

Als Methode zur Visualisierung und groben Konzeptualisierung diente das Design-Tool Figma, welches die Grundstruktur des Projekts skizzierte. Die genauen Funktionen wurden zu diesem Zeitpunkt noch nicht im Detail festgelegt.

Während der Umsetzungsphase traten Herausforderungen auf, insbesondere im Bereich des State-Managements. Die Standard-Features von Reacts State-Management erwiesen sich aufgrund zahlreicher Abhängigkeiten zwischen verschiedenen Komponenten als unzureichend. Als Lösung wurde die Integration der "Zustand"-Bibliothek gewählt, um eine effiziente Verwaltung des Anwendungsstatus zu ermöglichen.

Es gab Anpassungen an der ursprünglichen Planung, insbesondere im Bereich des State-Managements. Die anfängliche Absicht, Redux zu verwenden, wurde aufgrund von Lernschwierigkeiten durch "Zustand" ersetzt, um einen reibungslosen Entwicklungsprozess sicherzustellen.

Die Tests wurden als End-to-End (E2E) Tests mithilfe der Cypress-Bibliothek durchgeführt. Der Fokus lag dabei zunächst auf der Überprüfung der Kernfunktionalitäten, einschließlich der Texteingabe, Berechnung von Geschwindigkeiten und Steuerung der Animationen.

Die Dokumentation wurde kontinuierlich gepflegt, wobei nach Hinzufügen neuer Funktionen entsprechende Dokumentationsabschnitte hinzugefügt wurden. Diese praxisorientierte Dokumentationsstrategie gewährleistete Transparenz und Nachvollziehbarkeit im Entwicklungsprozess.

\section{Auftragsergebnis (Soll - Ist Vergleich)}

\subsection{Soll}
Die ursprüngliche Vision des Projekts war die Entwicklung eines textbasierten Rennspiels, das den Anforderungen des Simulationsprojekts entspricht und gleichzeitig ein unterhaltsames Spielerlebnis bietet. Die spezifischen Kriterien umfassten die Implementierung von Eingabeparametern, Zufallsparametern, verschiedenen Simulationsgeschwindigkeiten, einer Animation und einer zeitabhängigen Simulation.

\subsection{Ist}
Das resultierende Projekt ist ein funktionsfähiges textbasiertes Rennspiel, das erfolgreich die definierten Kriterien des Simulationsprojekts erfüllt. Die Oberfläche bietet mehr als fünf Eingabeparameter, darunter zwei Zufallsparameter in Form von Gauss'scher Normalverteilung und Normalverteilung. Drei unterschiedliche Simulationsgeschwindigkeiten (langsam, mittel, schnell) sowie eine Animation sind implementiert. Die Simulation ist zeitabhängig, und zusätzliche Elemente wie Barrieren sorgen für eine vielseitige und herausfordernde Spielerfahrung.

\subsection{Vergleich}
Der Vergleich zeigt eine weitgehende Übereinstimmung zwischen den ursprünglichen Zielen und dem erreichten Ergebnis. Die Funktionalitäten des Spiels erfüllen die gestellten Anforderungen und bieten darüber hinaus einen zufälligen und unterhaltsamen Aspekt durch die Implementierung von Items und Hindernissen. Die Simulation ist intuitiv und benutzerfreundlich, während die technische Umsetzung auf bewährten Technologien wie React, TypeScript und Zustand basiert.

\end{document}